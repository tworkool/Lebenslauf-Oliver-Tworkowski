\documentclass[11pt, a4paper]{moderncv}

\usepackage[utf8]{inputenc} 	% utf-8 encoding
\usepackage{graphicx} 			% Grafiken
\usepackage[ngerman]{babel} 	% deutsch als Hauptsprache
\usepackage[right]{eurosym} 	% Eurozeichen
\usepackage{fontspec} 			% Zeichenencoding
\usepackage{lmodern}			% flüssigere Schrift
\usepackage{longtable} 			% mehrseitige Tabellen
\usepackage[left=3cm,right=2.5cm,top=2.5cm,bottom=2.5cm,includeheadfoot]{geometry}
%\usepackage{wasysym}
								% Seitenränder
\usepackage{fancybox} 			% Paket für Boxen im Text
\usepackage{url} 				% bricht lange URLs "schön" um
\usepackage[table]{xcolor} 				% Textfarben
\usepackage{amsmath,amsthm,amssymb,amsfonts}
% \usepackage{hyperref} 		% erzeugt Inhaltsverzeichnis mit Querverweisen zu den Abschnitten (PDF Version)
\usepackage{fancyhdr} 			% Kopf- Fußzeilen
%\usepackage{array} 			% für Tabellen
\usepackage[onehalfspacing]{setspace} 
								% Paket für Zeilenabstand
\usepackage{caption} 			% extended captions
%\usepackage{makeidx} 			% extended index
\usepackage{listings} 			% code Blöcke
%\usepackage[backend=biber, sorting=none, citestyle=ieee]{biblatex}

%\usepackage[german]{nomencl}
%\usepackage{acronym}
%\usepackage{blindtext}
%\usepackage{csquotes}
%\usepackage[dvipsnames]{xcolor}
\usepackage{multirow}
\usepackage{float}
%\usepackage{dblfloatfix}
%\usepackage[titles]{tocloft} 	% Für Formeln und Formelverzeichnis
%\usepackage{floatflt} % floatende Bilder ermöglichen

% Ohne Einrückung bei neuem Absatz
\setlength{\parindent}{0pt}

% neue Kopfzeilen mit dem Paket fancyhdr
\pagestyle{fancy} %eigener Seitenstil
\fancyhf{} %alle Kopf- und Fußzeilenfelder bereinigen
%\fancyhead[L]{\nouppercase{\leftmark}} %Kopfzeile links
%\fancyhead[C]{} %zentrierte Kopfzeile
%%%\fancyhead[R]{\thepage} %Kopfzeile rechts
%\fancyfoot[R]{\thepage}
%%%\renewcommand{\headrulewidth}{0.4pt} %obere Trennlinie

\newcommand{\inq}[1]{
	\glqq{#1}\grqq{}
}

\definecolor{lightblue}{HTML}{156396}

\newcommand{\cvrating}[1]{
	\color{lightblue}
	\ifcase #1
			\square\square\square\square\square
		\or 
			\blacksquare\square\square\square\square
		\or 
			\blacksquare\blacksquare\square\square\square
		\or 
			\blacksquare\blacksquare\blacksquare\square\square
		\or 
			\blacksquare\blacksquare\blacksquare\blacksquare\square
		\or 
			\blacksquare\blacksquare\blacksquare\blacksquare\blacksquare
		\else 
			NONE!
			\GenericError{[mycode] }{Bad input: command cvrating only allows for input number from 0-5}%
			{[mycode] command cvrating only allows for input number from 0-5}%
			{}
	\fi
	\color{black}
}

\newcommand{\sfig}[1]{
	\begin{figure}
		\centering
		%\includegraphics[width=0.5\textwidth]{#1}
		\includegraphics[height=2cm, keepaspectratio=true]{#1}
	\end{figure}
}

\moderncvtheme[blue]{classic}
%\moderncvicons{awesome}

%Die persönlichen Daten:
\name{Oliver}{Tworkowski}
\title{Lebenslauf}
\address{Peter-Hille-Straße 33}{12587 Berlin}
\phone[mobile]{0176 43856069}
%\phone[fixed]{01234\,123456}
%\phone[fax]{01234\,012345}
\email{olitworkowski@gmail.com}
%\homepage{www.meine-homepage.de}
%\social[twitter][www.twitter.com]{Twitter-Name}
%\social[linkedin][www.linkedin.com]{LinkedIn-Name}
\social[github][www.github.com/tworkool/Lebenslauf-Oliver-Tworkowski]{tworkool}
%\extrainfo{Hier können extra Informaitonen stehen.}
\photo[3cm]{abb/Portrait_Oliver_Tworkowski}
%\quote{"Hier kann mein Lebensmotto oder ein Zitat stehen."}

\definecolor{tableRowColor}{HTML}{f0f0f0}
\begin{document}
	\rowcolors{1}{tableRowColor}{}
	\begin{longtable}[c]{c}
		% DIESE TABELLE WIRD NUR GEBRAUCHT UM \hiderowcolors ANZUWENDEN
		\hiderowcolors  
	\end{longtable}
	\makecvtitle
	
	\section{Persönliche Daten}
	\cvline{Geburtstag}{22. April 1998}
	\cvline{Geburtsort}{Berlin, Deutschland}
	
	\section{Schulausbildung}
	\cventry{2021--dato}{Universität}{Hochschule für Technik und Wirtschaft}{Berlin}{}{M. Sc. Angewandte Informatik}
	\cventry{2018--2021}{Universität}{Hochschule für Technik und Wirtschaft}{Berlin}{}{B. Sc. Ingenieurinformatik mit Endnote 1,5}
	\cventry{2010--2017}{Gymnasium}{Merian-Schule}{Berlin}{bilingualer Unterricht}{Abitur mit Endnote 2,1}
%	\cventry{2004--2010}{Grundschule}{Merian-Schule}{Berlin}{russisch Unterricht}{}
	
	\section{Sprachkenntnisse}
	\cvlanguage{deutsch}{Muttersprache}{}
	\cvlanguage{polnisch}{zweite Muttersprache}{}
	\cvlanguage{russisch}{erweiterte Grundkenntnisse}{Schulunterricht seit Grundschule}
	\cvlanguage{englisch}{hervorragende Kenntnisse}{Bilingualer Unterricht im Gymnasium}
	
	\section{Arbeitserfahrung}
	\cventry{Aug. 2019--Sep. 2021}{SAP SE}{}{Berlin}{}{Werkstudent im Team \inq{Identity and User Management}, Full Stack in Microservice-basierter Umgebung}
	\subsection{Pre-Bachelor}
	\cventry{2019}{Studienkreis GmbH}{}{Berlin}{}{Nachhilfelehrer Englischunterricht}
	\cventry{2018}{DeinKiez}{dreimonatiges Vollzeitpraktikum}{Berlin}{}{3D-Drucker Bedienung und Grafikdesign}
	\cventry{2017--2018}{Diverse Minijobs}{}{Berlin}{}{Unterschiedliche Jobs in der Gastronomiebranche und ein Bürojob}
	\cventry{2014--2017}{\inq{Stern-und Kreisschiffahrt} GmbH}{jeweils erste Juniwoche}{Berlin}{}{Tourguide für Sehenswürdigkeiten von Berlin}
	
	\section{Lizensen und Preise}
	\cventry{2020}{WWS Windsurflehrer Ausbildung}{}{}{}{}
	\cventry{26.11.2019}{Hackathon}{2. Platz}{StudySmarter}{}{Zum Thema: Demokratie am Campus}
	\cventry{Okt. 2017}{SBF Binnen Motor und Segel}{}{}{}{}
	\cventry{Feb. 2017}{Führerschein Klasse B}{}{}{}{}
	
	%\section{Publikationen}
	%\cvlistitem{Meine 1. Publikation.}
	%\cvlistitem{Noch eine Publikation.}
	
	\section{EDV- und Programmier-Kenntnisse}
	\begin{longtable}[c]{p{11cm} p{3cm}}
		%\caption{BLAAA}
		\label{tab:perf-2}\\
%		\bottomrule
		\textbf{Technologie} & \textbf{Kenntnis-Skala} \\
		%\midrule
		\hline
		\endfirsthead
		%
		\multicolumn{2}{c}{{\textit{Tabelle \thetable\ von letzter Seite fortgesetzt} }} \\
		\endhead
		%
		%\hline
		\multicolumn{2}{r}{\textit{Fortsetzung auf nächster Seite}} \\
		\endfoot
		%\hline
		\endlastfoot
		\showrowcolors
		%
		\LaTeX & $ \cvrating{4} $ \\
		C\# & $ \cvrating{4} $ \\
		C++ & $ \cvrating{4} $ \\
		Java & $ \cvrating{3} $ \\
		Androidentwicklung mit Kotlin & $ \cvrating{4} $ \\
		Python & $ \cvrating{5} $ \\
		Webentwicklung mit React JS* & $ \cvrating{5} $ \\
		SQL & $ \cvrating{3} $ \\
		MongoDB & $ \cvrating{3} $ \\
		Blender 3D & $ \cvrating{4} $ \\
		AutoCAD & $ \cvrating{4} $ \\
		Inventor & $ \cvrating{4} $ \\
%		\bottomrule 
		\hiderowcolors    
	\end{longtable}

* Zum Webentwicklungs-Stack gehört die Entwicklung einer responsive Webapp mit SASS, Parcel Bundler, Webpack Bundler, Redux JS und anderen kleineren JS Paketen.
	
	\newpage
	\section{Hobbys und Interessen}
	\cvlistdoubleitem{Windsurfen}{Ultimate Frisbee}
	\cvlistdoubleitem{Segeln}{Volleyball}
	\cvlistitem{Programmieren und Kennenlernen neuer Technologien}
	\cvlistdoubleitem{Events mit Freunden}{Videospiele}
	\cvlistitem{Musizieren und anderweitig künstlerisch Ausdrücken}
	
	\vfill  %Platz bis Seitenende auffüllen.
	\includegraphics[height=2cm, keepaspectratio=true]{abb/US_OT.png}\\
	Oliver Tworkowski\\
	Berlin, \today
	
%	\newpage
%	\section{Wassersportlicher Lebenslauf}
%	\cventry{ab 2018}{Windsurfen als Vollzeithobby}{}{}{}{Erstes, eigenes Windsurfmaterial gekauft, Fußschlaufen und Trapez-Fahren, Powerhalse, \inq{lightwind fundamentals}.}
%	\cventry{2017}{SBF Binnen}{}{Surf- und Segelschule Müggelsee, Berlin}{}{Sportboot-Führerschein für Binnengewässer unter Segel und Motor.}
%	\cventry{2017}{Segelregatta}{}{Almatur Giżycko (Polen)}{}{das gleiche wie im Vorjahr.}
%	\cventry{2016}{Segelregatta}{}{Almatur Giżycko (Polen)}{}{Zusammen mit meinem Onkel und 6 weiteren Seglern auf der Masurischen Seenplatte auf einem Zweimaster mit offenem Deck.}
%	\cventry{2012}{Erstes Mal windsurfen}{}{Surf- und Segelschule Müggelsee, Berlin}{}{}
%	
%	\vfill  %Platz bis Seitenende auffüllen.
%	\includegraphics[height=2cm, keepaspectratio=true]{abb/US_OT.png}\\
%	Oliver Tworkowski\\
%	Berlin, \today
\end{document}